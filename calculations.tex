\documentclass[linenumbers]{aastex62}

\begin{document}
\section{Accompanying calculations for Section 5.2.2.}

Magnetic energy density is given by $U'_B = B'^2 / 8\pi$, and total energy density of jet at radius $r$ is
$U'_{\rm jet} = L'_{\rm jet} / (4\pi c r^2) = L'_{\rm jet} / (4\pi c \Gamma^2 s^2)$, where $s = r / \Gamma$ is the the cross section of the jet with opening angle $\theta_\mathrm{jet} \sim \Gamma^{-1}$. Thus, 
\begin{equation}
 B' \sim \frac{2}{c} L_\mathrm{jet}^{\prime 1/2} s^{-1}\Gamma^{-1} \sim 30\,\mathrm{G}\,L_\mathrm{jet\,45}^{\prime1/2} s_{15}^{-1}\Gamma_{10}^{-1},
\end{equation}
where $L'_\mathrm{jet} =10^{45} \mathrm{erg}\,\mathrm{s}^{-1} L'_\mathrm{jet\,45}$, $s = 10^{15}\,\mathrm{cm}\,s_{15}$, and $\Gamma = 10\Gamma_{10}$.
The total energy density of the radiation (in SI units) is $U_r = (c^{-2}E^2 + B^2) / (2\mu_0)$. Assuming that the electric field is less or equal to the magnetic field, we get $E \leqslant cB$, where $B$ is in T and $\mathrm{T} = \mathrm{V}\,\mathrm{s}\,\mathrm{m}^{-2}$. Hence, 
\begin{equation}
E'_\mathrm{max} \sim 2 L_\mathrm{jet}^{\prime1/2} s^{-1}\Gamma^{-1} \sim 1L_\mathrm{jet\,45}^{\prime1/2} s_{15}^{-1}\Gamma_{10}^{-1}\,\mathrm{MV}\,\mathrm{m}^{-1},
\end{equation}
and the total potential difference across the jet is $V_\mathrm{max} \sim E'_\mathrm{max} r \sim 10^{16}E'_\mathrm{max} s_{15}\Gamma_{10}\sim 10^{16-2}L_\mathrm{jet\,45}^{\prime 1/2}\,\mathrm{MV} = 100L_\mathrm{jet\,45}^{\prime 1/2}\,\mathrm{EV}$.

Constraint on $R_{\rm blob}$: electromagnetic energy in plasma blob should be large enough to explain peak flare amplitude. 
From Roger's mail: 
$$ R_\mathrm{blob}^3 U_r \gtrsim 4\pi\Gamma^2 D^2 F $$. Is $D$ the luminosity distance and $F$ the energy flux in $\mathrm{erg}\,\mathrm{s}^{-1}\,\mathrm{cm}^{-2}$? Then the units don't match. 
\end{document}