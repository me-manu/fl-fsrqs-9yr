\documentclass[twocolumn]{aastex62}
\usepackage{xspace}

%% The default is a single spaced, 10 point font, single spaced article.
%% There are 5 other style options available via an optional argument. They
%% can be envoked like this:
%%
%% \documentclass[argument]{aastex62}
%% 
%% where the layout options are:
%%
%%  twocolumn   : two text columns, 10 point font, single spaced article.
%%                This is the most compact and represent the final published
%%                derived PDF copy of the accepted manuscript from the publisher
%%  manuscript  : one text column, 12 point font, double spaced article.
%%  preprint    : one text column, 12 point font, single spaced article.  
%%  preprint2   : two text columns, 12 point font, single spaced article.
%%  modern      : a stylish, single text column, 12 point font, article with
%% 		  wider left and right margins. This uses the Daniel
%% 		  Foreman-Mackey and David Hogg design.
%%  RNAAS       : Preferred style for Research Notes which are by design 
%%                lacking an abstract and brief. DO NOT use \begin{abstract}
%%                and \end{abstract} with this style.
%%
%% Note that you can submit to the AAS Journals in any of these 6 styles.
%%
%% There are other optional arguments one can envoke to allow other stylistic
%% actions. The available options are:
%%
%%  astrosymb    : Loads Astrosymb font and define \astrocommands. 
%%  tighten      : Makes baselineskip slightly smaller, only works with 
%%                 the twocolumn substyle.
%%  times        : uses times font instead of the default
%%  linenumbers  : turn on lineno package.
%%  trackchanges : required to see the revision mark up and print its output
%%  longauthor   : Do not use the more compressed footnote style (default) for 
%%                 the author/collaboration/affiliations. Instead print all
%%                 affiliation information after each name. Creates a much
%%                 long author list but may be desirable for short author papers
%%
%% these can be used in any combination, e.g.
%%
%% \documentclass[twocolumn,linenumbers,trackchanges]{aastex62}
%%
%% AASTeX v6.* now includes \hyperref support. While we have built in specific
%% defaults into the classfile you can manually override them with the
%% \hypersetup command. For example,
%%
%%\hypersetup{linkcolor=red,citecolor=green,filecolor=cyan,urlcolor=magenta}
%%
%% will change the color of the internal links to red, the links to the
%% bibliography to green, the file links to cyan, and the external links to
%% magenta. Additional information on \hyperref options can be found here:
%% https://www.tug.org/applications/hyperref/manual.html#x1-40003
%%
%% If you want to create your own macros, you can do so
%% using \newcommand. Your macros should appear before
%% the \begin{document} command.
%%
\newcommand{\vdag}{(v)^\dagger}
\newcommand\aastex{AAS\TeX}
\newcommand\latex{La\TeX}

\newcommand{\Gray}{$\gamma$~ray\xspace}
\newcommand{\Grays}{$\gamma$~rays\xspace}
\newcommand{\GRays}{$\gamma$~Rays\xspace}
\newcommand{\gray}{$\gamma$-ray\xspace}
\newcommand{\unit}[1]{\,\mathrm{#1}\xspace}
\newcommand{\Fermi}{\emph{Fermi}\xspace}
\newcommand{\FermiLAT}{\emph{Fermi}~LAT\xspace}
\newcommand{\fermiLAT}{\emph{Fermi}-LAT\xspace}
\newcommand{\todo}[1]{\textbf{\textcolor{red}{#1}}}

%% Reintroduced the \received and \accepted commands from AASTeX v5.2
\received{\today}
\revised{January 7, 2019}
\accepted{Soon}
%% Command to document which AAS Journal the manuscript was submitted to.
%% Adds "Submitted to " the arguement.
\submitjournal{ApJ}

%% Mark up commands to limit the number of authors on the front page.
%% Note that in AASTeX v6.2 a \collaboration call (see below) counts as
%% an author in this case.
%
%\AuthorCollaborationLimit=3
%
%% Will only show Schwarz, Muench and "the AAS Journals Data Scientist 
%% collaboration" on the front page of this example manuscript.
%%
%% Note that all of the author will be shown in the published article.
%% This feature is meant to be used prior to acceptance to make the
%% front end of a long author article more manageable. Please do not use
%% this functionality for manuscripts with less than 20 authors. Conversely,
%% please do use this when the number of authors exceeds 40.
%%
%% Use \allauthors at the manuscript end to show the full author list.
%% This command should only be used with \AuthorCollaborationLimit is used.

%% The following command can be used to set the latex table counters.  It
%% is needed in this document because it uses a mix of latex tabular and
%% AASTeX deluxetables.  In general it should not be needed.
%\setcounter{table}{1}

%%%%%%%%%%%%%%%%%%%%%%%%%%%%%%%%%%%%%%%%%%%%%%%%%%%%%%%%%%%%%%%%%%%%%%%%%%%%%%%%
%%
%% The following section outlines numerous optional output that
%% can be displayed in the front matter or as running meta-data.
%%
%% If you wish, you may supply running head information, although
%% this information may be modified by the editorial offices.
\shorttitle{Gamma-ray variability of bright FSRQs}
\shortauthors{Meyer et al.}
%%
%% You can add a light gray and diagonal water-mark to the first page 
%% with this command:
% \watermark{text}
%% where "text", e.g. DRAFT, is the text to appear.  If the text is 
%% long you can control the water-mark size with:
%  \setwatermarkfontsize{dimension}
%% where dimension is any recognized LaTeX dimension, e.g. pt, in, etc.
%%
%%%%%%%%%%%%%%%%%%%%%%%%%%%%%%%%%%%%%%%%%%%%%%%%%%%%%%%%%%%%%%%%%%%%%%%%%%%%%%%%

%% This is the end of the preamble.  Indicate the beginning of the
%% manuscript itself with \begin{document}.

\begin{document}

\title{Characterizing the gamma-ray variability of the brightest flat spectrum radio quasars observed with the \emph{Fermi} LAT}

%% LaTeX will automatically break titles if they run longer than
%% one line. However, you may use \\ to force a line break if
%% you desire. In v6.2 you can include a footnote in the title.

%% A significant change from earlier AASTEX versions is in the structure for 
%% calling author and affilations. The change was necessary to implement 
%% autoindexing of affilations which prior was a manual process that could 
%% easily be tedious in large author manuscripts.
%%
%% The \author command is the same as before except it now takes an optional
%% arguement which is the 16 digit ORCID. The syntax is:
%% \author[xxxx-xxxx-xxxx-xxxx]{Author Name}
%%
%% This will hyperlink the author name to the author's ORCID page. Note that
%% during compilation, LaTeX will do some limited checking of the format of
%% the ID to make sure it is valid.
%%
%% Use \affiliation for affiliation information. The old \affil is now aliased
%% to \affiliation. AASTeX v6.2 will automatically index these in the header.
%% When a duplicate is found its index will be the same as its previous entry.
%%
%% Note that \altaffilmark and \altaffiltext have been removed and thus 
%% can not be used to document secondary affiliations. If they are used latex
%% will issue a specific error message and quit. Please use multiple 
%% \affiliation calls for to document more than one affiliation.
%%
%% The new \altaffiliation can be used to indicate some secondary information
%% such as fellowships. This command produces a non-numeric footnote that is
%% set away from the numeric \affiliation footnotes.  NOTE that if an
%% \altaffiliation command is used it must come BEFORE the \affiliation call,
%% right after the \author command, in order to place the footnotes in
%% the proper location.
%%
%% Use \email to set provide email addresses. Each \email will appear on its
%% own line so you can put multiple email address in one \email call. A new
%% \correspondingauthor command is available in V6.2 to identify the
%% corresponding author of the manuscript. It is the author's responsibility
%% to make sure this name is also in the author list.
%%
%% While authors can be grouped inside the same \author and \affiliation
%% commands it is better to have a single author for each. This allows for
%% one to exploit all the new benefits and should make book-keeping easier.
%%
%% If done correctly the peer review system will be able to
%% automatically put the author and affiliation information from the manuscript
%% and save the corresponding author the trouble of entering it by hand.

\correspondingauthor{Manuel Meyer}
\email{mameyer@stanford.edu}

\author[0000-0002-0738-7581]{Manuel Meyer}
\affil{Kavli}

\author{Jeffrey Scargle}
\affil{NASA Ames}

\author{Roger Blandford}
\affil{Kavli}

\author{The \emph{Fermi}-LAT Collaboration}
\affiliation{Flying from a low Earth orbit}

%% Note that the \and command from previous versions of AASTeX is now
%% depreciated in this version as it is no longer necessary. AASTeX 
%% automatically takes care of all commas and "and"s between authors names.

%% AASTeX 6.2 has the new \collaboration and \nocollaboration commands to
%% provide the collaboration status of a group of authors. These commands 
%% can be used either before or after the list of corresponding authors. The
%% argument for \collaboration is the collaboration identifier. Authors are
%% encouraged to surround collaboration identifiers with ()s. The 
%% \nocollaboration command takes no argument and exists to indicate that
%% the nearby authors are not part of surrounding collaborations.

%% Mark off the abstract in the ``abstract'' environment. 
\begin{abstract}

Abstract

\end{abstract}

%% Keywords should appear after the \end{abstract} command. 
%% See the online documentation for the full list of available subject
%% keywords and the rules for their use.
\keywords{editorials, notices --- 
miscellaneous --- catalogs --- surveys}

%% From the front matter, we move on to the body of the paper.
%% Sections are demarcated by \section and \subsection, respectively.
%% Observe the use of the LaTeX \label
%% command after the \subsection to give a symbolic KEY to the
%% subsection for cross-referencing in a \ref command.
%% You can use LaTeX's \ref and \label commands to keep track of
%% cross-references to sections, equations, tables, and figures.
%% That way, if you change the order of any elements, LaTeX will
%% automatically renumber them.
%%
%% We recommend that authors also use the natbib \citep
%% and \citet commands to identify citations.  The citations are
%% tied to the reference list via symbolic KEYs. The KEY corresponds
%% to the KEY in the \bibitem in the reference list below. 

\section{Introduction} \label{sec:intro}

More than half of the sources observed with the \Fermi Large Area Telescope (LAT) above 100\,MeV are active galaxies that produce particle outflows (jets) at almost the speed of light, which are closely aligned to the line of sight.\footnote{See e.g. \url{http://www.asdc.asi.it/fermi3fgl/}}
%Yet, the exact mechanism and location of \gray production inside such jets of these so-called blazars remain a controversially discussed topic in the literature~\cite{Madejski:2016oqg}.
The broadband electromagnetic radiation observed from these so-called blazars spans decades in energy from radio frequencies up to very-high energy \gray energies. 
It is often described with purely leptonic or a mixture of leptonic and hadronic emission models, involving both intrinsic and external radiation fields~\cite[e.g.,][and references therein]{Madejski:2016oqg}.
%It is often explained with either purely leptonic or mixture of hadronic and leptonic emission models. 
%In leptonic models, the low energy part of the spectral energy distribution (SED) is caused by synchrotron emission of relativistic electrons in magnetic fields, whereas the high energy end of the SED is attributed to inverse Compton scattering of electrons with either the synchrotron emission or external radiation fields. 
%In leptonhadronic models, the high energy emission can be due to proton-synchrotron radiation, or the creation of particle cascades produced in interactions of cosmic rays with ambient gas and radiation fields. 
%In leptonic models, the emission is due to synchrotron emission and inverse Compton scattering of the electrons with synchrotron emission or external radiation fields. In leptohadronic models the high energy emission can be either due to proton-synchrotron radiation or emission produced in particle cascades, which are initiated by cosmic-ray interactions with gas and radiation fields.
A common assumption is that the radiation is emitted by freshly accelerated   particles localized in ``plasmoids''  that move down the jet at relativistic speeds,
 leading to a strong doppler boost of the observed emission. 
Yet, the origin and location of such plasmoids remain unknown. 

Blazars also display variable emission at all observationally accessible time scales, limited only by signal-to-noise.
Surprisingly, at \gray energies, flux doubling times as low as minutes have been observed both in BL Lac-type objects and flat spectrum radio quasars (FSRQs) with ground-based Cherenkov telescopes and the \FermiLAT~\cite[e.g.][]{pks2155hess2007,pks1222magic2011,TheFermi-LAT:2016dss}.
In these cases, causality arguments suggest extremely compact emission regions realized in, e.g., magnetic reconnection events or recollimation shocks~\cite[e.g.][]{Petropoulou:2016xat,Bodo:2017qqn}.
In particular for FSRQs, the observation of \Grays beyond 10\,GeV suggests that these compact dissipation sites are located at distances of hundreds of Schwarzschild  radii from the central super massive black hole. 
Otherwise, the \gray emission would be strongly attenuated in the interaction with UV and optical photons that are emitted from fast rotating clouds of ionized gas of the broad line region (BLR). 
Meeting these constraints is challenging for standard emission scenarios as extreme relativistic bulk motions of the plasma have to be invoked~\cite[e.g.,][]{TheFermi-LAT:2016dss}. 

After almost one decade of continuous all-sky observations, the \FermiLAT has accumulated a large sample of flares from many FSRQs.
Our goal is to characterize the flares and long-term behaviour of the FSRQs that have shown the brightest \gray flares over the course of the \Fermi mission. 
The brightest flares enable us to perform a comprehensive search for \gray variability on time scales as short as minutes in order to investigate if such short variability -- and conversely compact emission sites -- is a common phenomenon 
in FSRQ flares. 
Such short variability has already been discovered in 3C\,279~\citep{TheFermi-LAT:2016dss} and recently in CTA\,102~\citep{2018ApJ...854L..26S}, but searches in other sources have been unsuccessfull~\citep{2017Galax...5..100N}.
The plethora of observed flares also allows us to perform a systematic study of the local temporal flare profiles, which could be connected to particle injection or particle propagation.
\citet{2013MNRAS.430.1324N} investigated the brightest \gray flares blazar in the first 4 years of LAT data.
The author found that, on average, flares have a slight tendency towards rise times being shorter than decay times, however, no flare showed extreme asymmetry. 
\citet{2010ApJ...722..520A} characterized the blazars in terms of their power spectral density (PSD) using 11 months of data and found that bright blazars mostly fall in an intermittent regime between red noise (flickering) and Brownian noise. 
These analyses can be significantly extended with almost one decade of continuous \fermiLAT observations.


Additionally, the high signal-to-noise spectra during flaring states enable the search for spectral absorption features due to the interaction of \Grays with BLR photons.
The detection of such features would locate the \gray emission region inside the BLR with important implications where particles dissipate their energy. 
Indeed, evidence for such absorption was reported in early  \fermiLAT observations~\citep{2010ApJ...717L.118P}, but a recent analysis of a large sample of 100 FSRQs and over 7\,years of observations could not confirm this result~\citep{2018MNRAS.477.4749C}.
The absence of the absorption features can in turn be used to derive lower limits on the distance of the \gray emitting region to the central super-massive black hole. 

The article is organized as follows. 
In Sec.~\ref{sec:data} we present the source selection and \FermiLAT data analysis. 
We investigate the source behaviour on different time scales in Sec.~\ref{}, starting from light curves over the whole mission lifetime on consecutively zooming-in on \gray flares which we investigate on shorter time scales. 
For the first time, we use an unbiased method based on Bayesian blocks (BBs)~\citep{2013ApJ...764..167S} and one-dimensional group finding algorithms based on~\citet{1998ApJ...498..137E} to identify flares on shorter and shorter timescales.  
We discuss our findings in Sec.~\ref{}
before concluding in Sec.~\ref{}.

\section{Source selection and data analysis}
\label{sec:data}

We search for the brightest \gray flares among the sources included in the monitored source list\footnote{\url{https://fermi.gsfc.nasa.gov/ssc/data/access/lat/msl_lc/}}. 
Selecting blazars that have shown average daily fluxes $F \geqslant 10^{-5}\,\mathrm{cm}^{-2}\,\mathrm{s}^{-1}$ within ($1\,\sigma$ statistical uncertainties) above 100\,MeV,
we are left with a selection of 6 FSRQs, listed in Tab.~\ref{tab:src-select}, together with their coordinates, redshift, as well as black hole mass and luminosity of the $\mathrm{H}\beta$ line taken from the literature~\citep{2006ApJ...637..669L,2012RMxAA..48....9T}.  
The black hole mass and $\mathrm{H}\beta$ luminosity will be used in Sec.~\ref{} to model the optical line emission from the BLR.
With the chosen flux threshold we ensure that we have at least one flare of each source in the sample that is suitable to search for intra-orbit variability and to derive high signal-to-noise spectra. 
All of the selected FSRQs are well known \gray emitters and individual flares from these objects have been studied in great detail~\citep[e.g.,][]{}. 
As noted in the Introduction, two of the sources (3C\,279, CTA\,102) have already been shown to be variable on extremely short timescales~\citep{TheFermi-LAT:2016dss,2018ApJ...854L..26S}. 
Furthermore, PKS\,B1222+216, 3C\,279, and PKS\,1510-089  are among the 7 FSRQs also detected above 100\,GeV with imaging air Cherenkov Telescopes~\citep{}. 

\begin{deluxetable*}{llcccccc}
\tablewidth{0pt}
\tablecaption{ \label{tab:src-select}Blazars selected for this study. \todo{add Doppler, Gamma, obs angle from \citet{2017ApJ...846...98J}}}
\tablehead{Source name & 3FGL name & R.A. & DEC & Redshift & $\log_{10}(M_\bullet / M_\odot)$\tablenotemark{a} & $L_\mathrm{disk} [10^{46} \mathrm{ergs}\,\mathrm{s}^{-1}]$\tablenotemark{a} & $L(\mathrm{H}\beta) [10^{43} \mathrm{ergs}\,\mathrm{s}^{-1}]$\tablenotemark{b}}
\startdata
PKS\,B1222+216 & 3FGL\,J1224.9+2122 & 186.226  & 21.382 & 0.432 & 8.87\tablenotemark{c} & $1.612$ &  $2.788 \pm 0.561$\tablenotemark{d}\\
3C\,273 &	3FGL\,J1229.1+0202 & 187.266  & 2.051 & 0.158 & 	8.92 &	6.114 & 	15.40 \\
3C\,279 & 3FGL\,J1256.1-0547 & 194.045  & -5.786 & 0.5362 	&	8.28 &	1.110 &	1.728\\
PKS\,1510-089 &	3FGL\,J1512.8-0906 & 228.210  & -9.106 & 0.360 & 8.20 & 1.130 & 1.768\\
CTA\,102 & 3FGL\,J2232.5+1143 & 338.158  & 11.728 & 1.037 & 8.93\tablenotemark{c} & 3.996  &	8.929 $\pm$  6.000\tablenotemark{e}\\
3C\,454.3 & 3FGL\,J2254.0+1608 & 343.493  & 16.149 & 0.859 & 	8.83 &	7.186 & 	18.95 \\
\enddata
\tablenotetext{a}{Taken from \citet{2006ApJ...637..669L} if not noted otherwise.}
\tablenotetext{b}{Calculated from $L_\mathrm{disk}$ using Eq.~7 in \citet{2006ApJ...637..669L}.}
\tablenotetext{c}{From \citet{2014Natur.510..126Z}.}
\tablenotetext{d}{From \citet{2012RMxAA..48....9T}.}
\tablenotetext{e}{\citet{2012RMxAA..48....9T} give the $L$(CIV) with $(255.71 \pm 17.18)\times10^{43}\mathrm{ergs}\,\mathrm{s}^{-1}$ and Eq.~7 from \citet{2006ApJ...637..669L} is used to convert this to $L$(H$\beta$).}

\end{deluxetable*}

\subsection{Data selection}

Our goal is characterize both the long term \gray behaviour of the selected FSRQs as well as the brightest flares.
We therefore select \Grays that have been measured with the \FermiLAT between August 4, 2008, and January 30, 2018, yielding a total data set of almost 9.5\,years or 114\,months.
The \FermiLAT is a pair conversion telescope designed to measure \Grays with energies between 20\,MeV to above 300\,GeV~\citep{2009ApJ...697.1071A} and we 
 use \Grays within the energy range of 100\,MeV and 316\,GeV. 
Below 100\,MeV the effective area of the LAT quickly decreases and the point spread function increases to above $\sim 6^\circ$\footnote{See, e.g., \url{http://www.slac.stanford.edu/exp/glast/groups/canda/lat_Performance.htm}} making a point source analysis challenging. 
Since FSRQs usually have curved \gray spectra, we do not expect significant detection of these sources above our chosen maximum energy.
To mitigate contamination of \Grays of the Earth Limb, we limit the sample to events that have arrived at a zenith angle less than $90^\circ$ and we excise periods of bright GRBs and solar flares that have been detected with a test statistic $\mathrm{TS} > 100$.
The test statisitic is defined as $\mathrm{TS} = -2\ln(\mathcal{L}_1 / \mathcal{L}_0)$, i.e., the log-likelihood ratio between the the maximized likelihoods $\mathcal{L}_1$ and $\mathcal{L}_0$ for the hypotheses with and without an additional source, respectively~\citep{mattox1996}.
We use the latest \texttt{Pass 8} instrumental response functions and Monte Carlo simulations~\citep{pass8} and select \gray events that pass the \texttt{P8R2 SOURCE} event selection. 
For each source we analyze $10^\circ \times 10^\circ$ regions of interest (ROIs) centered on the position of each source as provided in the third \fermiLAT point source catalog \citep[3FGL,][]{3fgl}.
We choose a spatial binning of $0.1^\circ$ per pixel and 8 energy bins per decade. 

\subsection{ROI optimization}
\label{sec:roi}

Our analysis proceeds iteratively, starting from the full time range and zooming in on bright flares and shorter time scales (see Sec.~\ref{sec:zoom}).
In a first step, we optimize the global \gray model of each ROI using the \textit{Fermi Science Tools} version 11-05-03\footnote{\url{http://fermi.gsfc.nasa.gov/ssc/data/analysis/software}} and \textsc{fermipy}, version 0.16.0+188\footnote{\url{http://fermipy.readthedocs.io}}~\citep{fermipy}.
The initial model consists of all \gray point sources within $15^\circ$ from the ROI center included in the 3FGL as well as the standard templates for isotropic and Galactic diffuse emission.\footnote{For the Galactic diffuse emission we use the file gll\_iem\_v06.fits and the file iso\_P8R2\_SOURCE\_V6\_v06.txt for the isotropic diffuse component, see: \url{ http://fermi.gsfc.nasa.gov/ssc/data/access/lat/BackgroundModels.html}}
After an initial optimization, we free the spectral normalization of sources that are within $10^\circ$ from the ROI center or that are detected with $\mathrm{TS} > 50$.
The spectral shape parameters such as power-law indices, curvature, or cut-off energies are free to vary for sources within $5^\circ$ from the ROI center. 
We freeze all spectral source parameters for sources detected with $\mathrm{TS} < 1$ or if the number of predicted photons is less than $10^{-3}$.
The normalizations of the diffuse backgrounds are also left free during the fit, together with the spectral index of the Galactic diffuse background template.
After the fit has converged successfully, 
we relocalize the central \gray source and re-fit all spectral model parameters. The relocalized source positions are provided in Table~\ref{tab:src-select}.
After this step, we generate a $\mathrm{TS}$ map to search for additional point sources. For each pixel in the ROI, we add a putative point source with a power-law spectrum with index $\Gamma = 2$ and calculate its $\mathrm{TS}$. If  $\sqrt{\mathrm{TS}} \geqslant 5$, we permanently add the source at the position of the highest $\mathrm{TS}$ value and re-optimize the spectral parameters for the whole ROI. This step is repeated until no further sources are found.

With the best-fit model for each ROI, we compute the \gray light curves for the FSRQs with an initial binning of 7~days. 
In each light curve bin, we leave spectral parameters free during the fit for sources within $3^\circ$ from the ROI center and additionally the normalizations of the Galactic and isotropic emission. If any of these sources have $\mathrm{TS} < 1$ or the number of predicted photons is less than $10^{-3}$, all parameters are fixed to their average values.

\subsection{Zooming in on bright flares using an unbiased method to identify different activity states}
\label{sec:zoom}

The 9.5 year light curves for all considered FSRQs are shown in Fig.~\ref{fig:weekly}. If the source is detected with $\mathrm{TS} < 9$ within one time bin or the flux in one bin is equal or smaller than its statistical uncertainty $F_i \leqslant \sigma_i$, we show upper limits at the $2\,\sigma$ confidence level instead (open symbols). The average source fluxes with their 1$\,\sigma$ statistical uncertainties, $\overline{F} \pm \sigma_{\overline{F}}$, derived from the likelihood maximazation of the full time range are shown as gray bands. 
The flux measurements and uncertainties are used to derive BB representations of the light curves with the \textit{point measurement} algorithm of~\citet[][]{2013ApJ...764..167S}, which are shown as orange lines. 
The BBs provide an unbiased way to detect significant local variations in the light curve.
Several strong flares exceeding the average flux level are easily identified from the BBs. 

%%% Full light curve %%%
\begin{figure*}
    \centering
    \includegraphics[width = .9\linewidth]{figures/lc_weekly_tsmin9.pdf}
    \caption{\gray light curves with weekly binning for the considered FSRQs. Open symbols denote upper limits at the $2\,\sigma$ confidence level. The orange lines show the BBs and the shaded regions represent the identified HOP groups. The green shaded regions denote the time itnervals identified as large flares for which the zoomed-in analysis is performed.}
    \label{fig:weekly}
\end{figure*}

There is no generally accepted consensus to determine the data points that belong to a flaring state and which characterize the quiescent level. \citet{2013MNRAS.430.1324N} suggests a simple definition that a flare is a continuous time interval associated with a flux peak in which the flux is larger than half the peak flux value. 
This definition is intuitive, however, it is unclear how to treat overlapping flares and identify flux peaks in an unbiased way. To mitigate these problems, we have implemented one dimensional implementation of the HOP algorithm~\citep{1998ApJ...498..137E}.
The algorithm associates data points to their nearest neighbor weighted by flux.
We feed the BB representation of the light curve to the HOP algorithm and the resulting groups of data points are shown 
Fig.~\ref{fig:weekly} as left and right hatched areas. \todo{some more details}
The combination of BBs and the HOP algorithm represents an unbiased way to split a light curve into groups of quiescent and flaring episodes; we will refer to one connected flare episode as a \emph{HOP group} of consecutive BBs.

We iteratively zoom in on time ranges with bright \gray activity by identifying HOP groups where one BB fulfills the condition $F_{BB} \geqslant F_\mathrm{max} =  5\times\overline{F}$ and including adjacent blocks with $F_{BB} \geqslant F_\mathrm{min} = \overline{F}$. We prefer this minimum flux criterion over including the full HOP group as it might extend over long time ranges of a quiescent state.
Overlapping time ranges are combined into one time range and the ranges are extended by one time bin on each side. Each time range is extended by one time bin on either side.
For the identified time span, we re-optimize the spectral model of the ROI in the same way as described in Sec.~\ref{sec:roi} but without re-localizing the central FSRQ or adding new point sources. Subsequently, we calculate a light curve with finer binning and again select the time ranges of the highest \gray activity. We repeat this procedure twice, down to a binning equal to the Good Time Intervals (GTIs), of the \Fermi satellite, which correspond to one passage of the source through the field of view of the satellite during one $\sim$ 95 minute orbit.
The choices of time binnings and values for $F_\mathrm{max}$ and $F_\mathrm{min}$ are summarized in Table~\ref{tab:zoom} together with the number of identified high \gray activity states (which might consists of several flares as indicated by the HOP groups).
The values of the threshold fluxes $F_\mathrm{max}, F_\mathrm{min}$ is somewhat arbitrary and are a compromise between including as many flares as possible but keeping the overall number of flares manageable. Note, that the sole purpose of this exercise is to select the brightest flares for  further analysis. 
The resulting time ranges for the weekly light curves are plotted as green shaded areas in Fig.~\ref{fig:weekly}.
The intermittent daily light curves are provided in Fig.~\ref{fig:daily} and  Fig.~\ref{fig:gti} shows the GTI light curves. 
The source exposure can vary significantly between two adjacent orbits, as the satellite rocks between the celestial north and south pole in between orbits. This explains the large error bars on some of the time bins of the GTI light curves.

In a last step, we derive light curves on sub-GTI time scales. 
The time bin size is calculated from the adaptive binning method of \citet{lott2012}, where we choose bins of constant flux uncertainty of $\sim20\,\%$. 
In this step, we use the space craft information in time steps of 1\,s instead of 30\,s. Additionally, we compute the effective area in 5 bins of the azimuthal spacecraft coordinates since on such short time scales the exposure dependence on the azimuth cannot be averaged over.

\begin{deluxetable}{cccc}
\tablewidth{1\linewidth}
\tablecaption{ \label{tab:zoom} Thresholds for BB fluxes in one HOP group to select time ranges of \gray activity together with selected time binning and number of selected time ranges.
If no interval fulfills the $F_\mathrm{BB} \geqslant F_\mathrm{max}$ criterion but flux points exceed $5\times10^{-6}\,\mathrm{cm}^{-2}\,\mathrm{s}^{-1}$ we change $F_\mathrm{max}$ to the maximum value $F_\mathrm{BB}$.}
\tablehead{Binning & $F_\mathrm{min}$ & $F_\mathrm{max}$ & $N_\mathrm{time~ranges}$}
\startdata
7 days & $\overline{F}$ & $5\times\overline{F}$ & 20\\
1 day & $\overline{F}$ & $\mathrm{max}(10^{-5}\,\mathrm{cm}^{-2}\,\mathrm{s}^{-1}, 1.5 \times \overline{F})$\tablenotemark{a} & 21\\
GTI & $\overline{F}$ & $2\times\overline{F},~\mathrm{TS} \geqslant 150$\tablenotemark{b} & 7\\
%Sub-GTI & -- & -- & --\\
\enddata
\tablenotetext{a}{We choose here the absolute flux (rather than the flux relative to the average) as a threshold in order to be consistent with our initial source selection. However, because of the high avergage flux of 3C454.3, we also include the max argument. If $F_\mathrm{max} = 1.5 \times \overline{F}$ we set $F_\mathrm{min} = 10^{-5}\,\mathrm{cm}^{-2}\,\mathrm{s}^{-1}$. }
\tablenotetext{b}{Motivated from the high $\mathrm{TS}$ found for the flare of 3C\,279~\citep{TheFermi-LAT:2016dss}, we also demand that at least one GTI of each HOP group is detected with $\mathrm{TS} \geqslant 150$ in order to ensure enough statistics to search for variability on time scales of minutes.}
\end{deluxetable}


%%% Daily light curves %%%%%%%%%
\begin{figure*}
    \centering
    \includegraphics[width = .99\linewidth]{figures/lc_daily_tsmin9.pdf}
    \caption{\label{fig:daily} Light curves with daily binning for the selected time ranges (green shaded regions in Fig.~\ref{fig:weekly}). Symbols and lines are the same as in Fig.~\ref{fig:weekly}.}
\end{figure*}
%%%%%%%%%%%%%%%%%%%%%%%%%%%%%%%%


%%% GTI light curves %%%%%%%%%
\begin{figure*}
    \centering
    \includegraphics[width = .99\linewidth]{figures/lcfithop_orbit_all_maxiter2_fsys0p00_addcomp0_comb.pdf}
    \caption{ Light curves with one bin per GTI for the selected time ranges (green shaded regions in Fig.~\ref{fig:daily}). Solid black and dashed lines show fits to the light curve with exponential flare profiles discussed in Sec.~\ref{sec:hop-fit}. Other symbols and lines are the same as in Fig.~\ref{fig:weekly} and ~\ref{fig:daily}.}
    \label{fig:gti}
\end{figure*}
%%%%%%%%%%%%%%%%%%%%%%%%%%%%%%

\section{Results for global light curve properties}
\label{sec:results-global}
We first present results derived from the weekly \gray light curves spanning the full 9.5\,year time range, which we refer to \emph{global light curve properties}, before deriving results from the local light curves on GTI and sub-GTI time scales in Sec.~\ref{sec:results-local}.

From the weekly light curves in Fig.~\ref{fig:weekly} it is evident that the FSRQs show strong flares that exceed the average flux by a factor of a few, while the quiescent level is relatively stable. 
Such behavior is typical for FSRQs~\citet{} and we 
further quantify it by calculating the flux distribution, $dN/dF$, of the weekly fluxes for bins with $\mathrm{TS} > 9$ and $F_i > \sigma_i$. 
The results are shown in Fig.~\ref{fig:fluxpdf}. The flux bins are chosen according to the algorithm of~\citet{knuth2006} and the error bars are calculated under the assumption that the observed weekly fluxes, $F_i$, $i = 1,\ldots,N$, are Gaussian distributed numbers with standard deviation equal to the measurement uncertainty $\sigma_i$.\footnote{
With this assumption, the uncertainty to find $x$ entries in the $j$-th flux bin of width $\Delta F_j = F_{\mathrm{hi},j} - F_{\mathrm{lo},j}$ is given by the sum of Bernoulli probabilities $p_{ij}$, $\sum_{i = 1}^N p_{ij}(1-p_{ij})$, where $p_{ij} =  \left[\mathrm{erf}\left((F_{\mathrm{hi},j} - F_i) / \sqrt{2\sigma_i^2}\right) - \mathrm{erf}\left(((F_{\mathrm{lo},j} - F_i) / \sqrt{2\sigma_i^2}\right)\right]/2$, and $\mathrm{erf}$ is the error function.
}
We fit the flux distribution with a smoothly broken power law (BPL) of the form 
\begin{equation}
    \frac{dN}{dF} = N_0 \left( \frac{F}{F_0}\right)^{\alpha_\mathrm{low}}
        \left( 1 - \left(\frac{F}{F_\mathrm{br}}\right)^s \right)^{\frac{\alpha_\mathrm{high} - \alpha_\mathrm{low}}{s}},
        \label{eq:dndf}
\end{equation}
with the smoothing factor $s$ fixed to 3. 
The results of a $\chi^2$ minimazation are summarized in Tab.~\ref{tab:global}.
Generally, below the break flux $F_\mathrm{br}$, the flux distribution is flat, $\alpha_\mathrm{low}\sim 0$.
Above $F_\mathrm{br}$ which lies between $\sim2\times10^{-7}$ and $\sim2\times10^{-6}\,\mathrm{cm}^{-2}\mathrm{s}^{-1}$, $dN/dF$ declines exponentially with power-law indices $\alpha_\mathrm{high} \lesssim -2.2$, making the brightest flares rare events.
Furthermore, it is clear that the flux distribution is very different from Gaussian behavior. 

\begin{figure}
    \includegraphics[width = .99\linewidth]{figures/fluxdist_weekly_tsmin9.pdf}
    \caption{\label{fig:fluxpdf} Distribution of the fluxes of the weekly 9.5 year \gray light curves. The BPL fit is shown as a black line with $1\,\sigma$ uncertainties shown as shaded bands.}
\end{figure}

We further characterize the global \gray light curves in terms of their PSD,
which usually can be described with simple power laws in frequency,  $\mathrm{PSD} \propto 1 / \nu^\beta$~\citep{}.
An analysis of the first 11~months of LAT data from 106 blazars revealed that 
these objects have $\beta$ values between 1 and 2, the intermittent regime between flicker noise ($\beta = 1$) and Brownian motion ($\beta = 2)$~\citep{2010ApJ...722..520A}. 
In addition to the noise behavior, 
we further use the derived PSDs in Sec.~\ref{} to simulate \gray light curves in order to calculate the significance of a correlation between radio and \gray emission. 

The best-fit PSDs are estimated from the periodograms and simulated light curves following the method described in detail in \citet{2014MNRAS.445..437M} and \citet{2013MNRAS.433..907E} and summarized briefly below.
The observed periodograms $P(\nu)$ as a function of frequency $\nu$ are calculated from the absolute square of the Fourier transformation of the light curve (Eq.~(3) of \citealt{2014MNRAS.445..437M}) including all 
data points detected with $\mathrm{TS} \geqslant 9$ and performing a linear interpolation between gaps in the light curve. Since we are using weekly binned light curves and bright FSRQs the gaps are small and at most 6 consecutive data points long (42\,days) in the case of PKS\,B1222+216. The number of non-detected bins is less than $\sim13\,\%$ for all sources.
The interpolation is done in time steps of $\Delta t=0.7\,$days and 
 re-binned into bins of lengths of 7 days by taking the geometrical mean of the flux.
In contrast to~\citet{2014MNRAS.445..437M}, we do not apply a window function (see the discussion below). 

We simulate light curves using the method of \citet{1995A&A...300..707T} with a time steps equal to 0.7\,days for power-law PSDs with values $0 \leqslant \beta \leqslant 3$ in steps of $\Delta\beta = 0.05$. For each $\beta$ value 100 light curves are generated, each one a 100 times longer than the actual observation to account for possible red-noise leakage. Splitting the simulated light curves (without overlap) leaves us with $10^4$ realizations. 
The light curves are then re-binned into 7-day light curves through averaging. The same observational gaps and interpolation as in the observed light curves are applied. 
The periodograms are then calculated for light curve in the same way for the observed light curve.
To fix the normalization of the PSD model, \citet{2014MNRAS.445..437M} suggest variance matching, i.e., they rescale the simulated flux data points with a factor $A^{-1}$, where $A^2 = \sigma_\mathrm{sim}^2 / (\sigma_\mathrm{obs}^2 - \bar{\sigma_i^2})$, with $\sigma_\mathrm{sim}^2$ ($\sigma_\mathrm{data}^2$) the variance of the simulated (observed) light curve and $\bar{\sigma_i^2}$ the variance of the observational noise.
For the \gray light curves, we choose to follow \citet{2013MNRAS.433..907E} instead and iteratively match the probability distribution of the simulated fluxes to the observed ones, given by the $dN/dF$ distrutions shown in Fig.~\ref{fig:fluxpdf}. 
The reason is that the algorithm of \citet{1995A&A...300..707T} produces light curves with Gaussian distributed fluxes, which is clearly not the case at \gray energies.\footnote{Furthermore, the variance matching relies on Parseval's theorem from which it follows that the light curve variance is equal to the integrated PSD. However, Parseval's theorem is only valid for square-integrable functions, i.e. $\beta \geqslant 2$ and thus not strictly applicable for smaller values of $\beta$ commonly observed at \gray energies.}
In a final step, we add uncertainties to the light curves by randomly drawing with replacement from the observed uncertainties $\sigma_i$ and adding a Gaussian random number $\mathcal{N}(0,\sigma_i)$ to the simulated flux values.

The peridograms of the observed and simulated light curves, ${P}_\mathrm{obs}$ and ${P}_\mathrm{sim}$, are averaged in logarithmic bins~\citep{1993MNRAS.261..612P} and compared by means of a $\chi^2$ test~\citep{2014MNRAS.445..437M},
\begin{equation}
    \chi^2(\beta) = \sum_{\nu_\mathrm{min}}^{\nu_\mathrm{max}}\frac{(P_\mathrm{obs}(\nu) - \overline{P}_\mathrm{sim}(\nu,\beta))^2}{\Delta\overline{P}_\mathrm{sim}(\nu,\beta)^2},\label{eq:chi2psd}
\end{equation}
where $\Delta\overline{P}_\mathrm{sim}(\nu)^2$ is the variance of the simulated light curves.
The averaged periodograms of the simulated light curves and the observed ones are shown in Fig.~\ref{fig:periodograms} and the best fit average periodogram is shown as a thick solid line. 
The quality of the the best-fit value $\hat\beta$ with corresponding minimum $\chi^2$ value $\hat\chi^2\equiv\chi^2(\hat\beta)$, is evaluated from the light curves simulated with $\beta = \hat\beta$ in the following way. We form the distributions of simulated $\chi^2$ values, $\chi^2_\mathrm{sim}$, by replacing $P_\mathrm{obs}(\nu)$ with $P_\mathrm{sim}(\nu,\beta)$ in Eq.~(\ref{eq:chi2psd}),
\begin{equation}
    \chi^2_\mathrm{sim}(\beta,\beta') = \sum_{\nu_\mathrm{min}}^{\nu_\mathrm{max}}\frac{(P_\mathrm{sim}(\nu,\beta) - \overline{P}_\mathrm{sim}(\nu,\beta'))^2}{\Delta\overline{P}_\mathrm{sim}(\nu,\beta')^2},\label{eq:chi2psd_sim}
\end{equation}
and calculate the $p$-value as the  fraction of simulations that result in $\chi^2_\mathrm{sim}(\hat\beta,\hat\beta) > \hat\chi^2$.
The confidence interval for $\hat\beta$ is derived by determining the $\Delta\chi^2_\mathrm{sim}(\hat{\beta},\beta)$ value from simulations such that 95\,\% of the time the simulated (true) $\beta$ value is contained within $\Delta\chi^2$. 
The same $\Delta\chi^2$ value is then applied to the observed $\chi^2$ curve.
The results of our PSD analysis are summarized in Tab.~\ref{tab:global} where we also report the value of $\beta$ obtained from a linear regression in log-log space. 
In general, the periodograms are well fit by our method, as indicated by the $p$-values and observed in Fig.~\ref{fig:periodograms}. 
The only exception is 3C\,279 where only 2 of the $10^4$ simulated light curves result in a $\chi^2_\mathrm{sim}(\hat\beta,\hat\beta) > \hat\chi^2$.
The steep $\chi^2$ curve for this source also explains the small error bars on the reconstructed value of $\beta$.
The reason might be a more complex underlying PSD or the specific 7 day binning we have chosen here. 
\todo{mention bumps / QPOs?}

\begin{figure*}
    \centering
    \includegraphics[width = 0.32\linewidth]{figures/periodogram_fermi_PKSB1222+216_Nsim_100Next_100Sim_addunc_data_rescale_EM13_usegap_1_PSD_window_none_detrend_none_norm_var_20.pdf}
    \includegraphics[width = 0.32\linewidth]{figures/periodogram_fermi_3C273_Nsim_100Next_100Sim_addunc_data_rescale_EM13_usegap_1_PSD_window_none_detrend_none_norm_var_20.pdf}
    \includegraphics[width = 0.32\linewidth]{figures/periodogram_fermi_3C279_Nsim_100Next_100Sim_addunc_data_rescale_EM13_usegap_1_PSD_window_none_detrend_none_norm_var_20.pdf}
    \includegraphics[width = 0.32\linewidth]{figures/periodogram_fermi_PKS1510-089_Nsim_100Next_100Sim_addunc_data_rescale_EM13_usegap_1_PSD_window_none_detrend_none_norm_var_20.pdf}
    \includegraphics[width = 0.32\linewidth]{figures/periodogram_fermi_CTA102_Nsim_100Next_100Sim_addunc_data_rescale_EM13_usegap_1_PSD_window_none_detrend_none_norm_var_20.pdf}
    \includegraphics[width = 0.32\linewidth]{figures/periodogram_fermi_3C454p3_Nsim_100Next_100Sim_addunc_data_rescale_EM13_usegap_1_PSD_window_none_detrend_none_norm_var_20.pdf}
    \caption{Periodograms of the observed (markers) and simulated light curves (colored lines). The simulated periodograms follow power-law PSDs between $\beta = 0$ (purple line) to $\beta = 3$ (red line) in steps of $\Delta\beta = 0.05$. The bottom panels show the residuals with respect to the best fit which is indicated in the legend and as a thick solid line in the upper panels.}
    \label{fig:periodograms}
\end{figure*}


\begin{deluxetable*}{lcccccc}
\tablewidth{0pt}
\tablecaption{ \label{tab:global}Global \gray light curve properties.}
\tablehead{Source name & $\alpha_\mathrm{low}$ & $\alpha_\mathrm{high}$ & $F_\mathrm{br} [10^{-6}\mathrm{cm}^{-2}\mathrm{s}^{-1}]$ 
& $\beta_\mathrm{slope}$ & $\hat\beta$ & $p$-value
}
\startdata
PKSB1222+216 & $-0.24^{+0.41}_{-0.27}$ & $-2.70^{+0.33}_{-0.43}$ & $0.42^{+0.28}_{0.15}$ &  1.23 & $1.12^{+0.21}_{-0.26}$ & 0.423 \\
3C273 & $0.70^{+0.40}_{-0.54}$ & $-2.77^{+0.24}_{-0.27}$ & $0.23^{+0.07}_{0.07}$  & 1.14 & $1.09^{+0.24}_{-0.27}$ & 0.330 \\
3C279 & $0.68^{+0.27}_{-0.40}$ & $-2.80^{+0.21}_{-0.23}$ & $0.40^{+0.10}_{0.10}$ & 0.67 & $0.61^{+0.10}_{-0.07}$ & $2\times10^{-4}$ \\
PKS1510-089 & $0.84^{+0.72}_{-0.48}$ & $-2.21^{+0.23}_{-0.26}$ & $0.40^{+0.16}_{0.12}$ & 0.88 & $1.00^{+0.18}_{-0.25}$ & 0.129 \\
CTA102 & $-0.35^{+0.22}_{-0.18}$ & $-2.50^{+0.16}_{-0.19}$ & $0.90^{+0.38}_{0.26}$ & 1.21 & $1.18^{+0.16}_{-0.32}$ & 0.138 \\
3C454.3 & $-0.20^{+0.17}_{-0.13}$ & $-3.00^{+0.13}_{-0.14}$ & $2.19^{+0.39}_{0.32}$ & 1.05 & $1.15^{+0.32}_{-0.28}$ & 0.274 \\
\enddata
{
\tablecomments{Columns 2-4 indicate the best-fit values for the BPL fit [Eq.~(\ref{eq:dndf})] to the $dN/dF$ distributions, whereas columns 5-7 show the best-fit results for the PSD. $\beta_\mathrm{slope}$ gives the result for a linear regression of the periodograms and $\hat\beta$ is the best-fit value of the $\chi^2$ minimization with corresponding $p$-value. The interval around $\hat\beta$ is at 95\,\% confidence.}
}
\end{deluxetable*}

\section{Results for local light curve properties}
\label{sec:results-local}
%\subsection{Local light curve properties}
%\label{sec:hop-fit}

We proceed with deriving local properties of the \gray flares from the light curves with one bin per GTI that are shown in Fig.~\ref{fig:gti}.

To assess the time profile of the flares,
we fit each HOP group $i$ with a $\chi^2$ minimization as a sum of exponential profiles, 
\begin{eqnarray}
    F_{\mathrm{flare},i}(t) &=& 
    \sum\limits_{j = 1}^{N_i} F_{0,ij}\nonumber\\
    &\times&\left[\exp\left(\frac{t - t_{0,ij}}{\tau_{\mathrm{rise},ij}}\right) + \exp
    \left(\frac{t_{0,ij} - t}{\tau_{\mathrm{decay},ij}}\right)\right]^{-1}\!\!\!,
    \label{eq:flareHOP}
\end{eqnarray}
where $t_{0,j}$ are the flare peak times, and $\tau_{\mathrm{rise},ij}$, $\tau_{\mathrm{decay},ij}$ are the flare rise and decay times, respectively.
All light curve points are included that fulfill $\mathrm{TS}\geqslant9$ and $F_i \geqslant 3\sigma_i/2 $.
The number of flare profiles per HOP shed, $N_i$, is either 1 or 2 and determined during the fit using the Bayesian information criterion (BIC), defined as $\mathrm{BIC} = n_\mathrm{par}\ln(n) + \chi^2$, where $n_\mathrm{par}$ is the number of fit paramteres ($n_\mathrm{par} = 4$ for $N_i = 1$), $n$ is the number of data points within one HOP group $i$. Two flare profiles are selected if the difference between the two BIC values is $\Delta\mathrm{BIC} = \mathrm{BIC}(N_i = 2) - \mathrm{BIC}(N_i = 1) < 0$.
The reason for allowing $N_i > 1$ is that the flare profile in Eq.~(\ref{eq:flareHOP}) does not capture long-lasting plateaus of a flare present in multiple flares in Fig.~\ref{fig:gti} (see, e.g., all flares of 3C\,279 or the panel with a flare of 3C\,454.3 starting at 55551.65\,MJD).

After each HOP group is fitted individually and $N_i$ is determined, 
we re-fit the entire light curve, which consists of $N_\mathrm{HOP}$ groups, with the function 
\begin{equation}
    F_\mathrm{flare}(t) = \sum\limits_{i = 1}^{N_\mathrm{HOP}}F_{\mathrm{flare},i}(t) + F_\mathrm{bkg}(t),
\end{equation}
where $F_\mathrm{bkg}(t)$ is a order-2 polynomial to describe a slow  varying background.
The fit results are shown as black solid lines in Fig.~\ref{fig:gti}.
In general, the $\chi^2$ values divided by the degrees of freedom (dof) are between 1 and 2 (see the legends in Fig.~\ref{fig:gti}). Given the large values of dof, the fit qualities are usually poor. This is not unexpected, as we only allow up to two flare profiles per HOP group and not arbitrary functions. Already with this choice, there are probably some spurious flares identified, see, e.g., the second flare profile in the first PKS\,1510-089 flare (starting at 54908.65MJD). 
Nevertheless, the overall light curve evolution is well captured, which allows us to describe the local flare properties from the ensembles of flare profiles, keeping the above caveats in mind. 

We show the distribution of rise and decay times in the histograms of Fig.~\ref{fig:times-hist} for flares with a time integrated flux $>10^{-7}\,\mathrm{cm}^{-2}$. 
Remarkably, all sources show values of either $\tau_\mathrm{rise}$ or $\tau_\mathrm{decay}$ or both that are below the horizon crossing time scale of the central super massive black hole in the observer's frame, $t_g = r_g / c$, with $r_g = 2 G M_\bullet / c^2 $ the gravitational radius, $G$ the gravitational constant, black hole mass $M_\bullet$ (taken from Tab.~\ref{tab:src-select}), and $c$ the speed of light.
Most flares rise and decay within 2 hours or less. 
\begin{figure}
    \centering
    \includegraphics[width = .8\linewidth]{figures/lcfithop_results_tdtrhist_maxiter2_fsys0p00_addcomp0_orbit.pdf}
    \caption{Stacked bar graphs for  rise (top) and decay times (bottom) for the individual flares fitted in Fig.~\ref{fig:gti}. The legend gives the horizon crossing time scales $t_g$ in the observer's frame.}
    \label{fig:times-hist}
\end{figure}

From the rise and decay times, we can calculate the flare asymmetry as
\begin{equation}
    A = \frac{\tau_\mathrm{rise}-\tau_\mathrm{decay}}
    {\tau_\mathrm{rise}+\tau_\mathrm{decay}},
\end{equation}
so that $A < 0$ for fast-rise-exponential-decay (FRED) type flares, as expected from an injection of energetic particles that subsequently cool through radiative processes such as inverse Compton scattering or synchrotron emission.
The asymmtry is shown versus integrated flux, the peak flux, and the flare duration $T_{90}$, defined in the time around the flare peak that contains 90\,\% of the integrated flux, in Fig.~\ref{fig:asym}.
The peak flux for each flare of each HOP group is derived from the maximum of  Eq.~(\ref{eq:flareHOP}) with respect to time (suppressing indeces),
\begin{equation}
    F_{\mathrm{peak}} = \frac{F_{0} \tau_\mathrm{rise}}{\tau_\mathrm{rise} + \tau_\mathrm{decay}}\exp\left(\ln\left(\frac{\tau_\mathrm{decay}}{\tau_\mathrm{rise}}\right)\frac{\tau_\mathrm{decay}}{\tau_\mathrm{rise} + \tau_\mathrm{decay}}\right)
\end{equation}
The error bars on the peak flux and asymmetry are derived from standard Gaussian error propagation from the fit uncertainties.
\begin{figure*}
    \centering
    \includegraphics[width = .8\linewidth]{figures/lcfithop_results_asym_vs_all_orbit_maxiter2_fsys0p00_addcomp0.pdf}
    \caption{Flare asymmetry versus integrated flux (left), peak flux (center), and flare duration $T_{90}$ (right) for the fitted flare profiles shown in Fig.~\ref{fig:gti}.}
    \label{fig:asym}
\end{figure*}

The median of the asymmetry is found to be $-0.195$, i.e., FRED-type flares are more common than the opposite. In general, the flares show a versatile behaviour and no clear trends are seen from Fig.~\ref{fig:asym}. This is also reflected in the fact that we do not find any significant correlation between the asymmetry, integrated and peak flux, rise and decay times as well as flare duration using Kendall's $\tau$.

We also investigate whether subsequent flares in each panel of Fig.~\ref{fig:gti} show a trend with time in peak flux, asymmetry, or duration. 
For consecutive flares, we calculate the difference between, e.g., the peak fluxes, and calculate the $p$-value of a Binomial distribution assuming an equal probability of finding negative and positive differences.
For 32 values of differences the $p$-values for the peak flux, asymmetry, as well as for the flare duration are close to 0.1 (14, 13, and 13 positive values for 32 trials, respectively) indicating no particular evolution of these quantities with time. 

We also find complex behavior of the spectral evolution during the flares. The often observed trend ``harder-when-brighter'' is found  for some sources but not in a strict manner. 
We therefore cannot draw any firm conclusions from the spectral evolution, which we show for reference in Fig.~\ref{fig:specvar} in  Appendix~\ref{sec:specvar}. 



\subsection{Sub-GTI light curves}
As described in Sec.~\ref{sec:zoom}, we search for sub-orbital variability in a subset of orbital light curves. 
We choose only those light curves where at least one orbital bin is detected with $\mathrm{TS} \geqslant 150$. 
In this way we ensure similarly high photon statistics and reduce the number of trials when searching for minute scale variability (for comparison, the orbital light curve bin for which \citet{TheFermi-LAT:2016dss} measured minute scale variability is detected with \todo{$TS = XXX$}). 
The selected time regions are indicated with the green shaded regions in Fig.~\ref{fig:gti}, whereas the blue shaded regions show the time intervals selected with the criteria in Tab.~\ref{tab:zoom} that do not pass the additional $\mathrm{TS}$ cut.

The resulting light curves, binned such that the uncertainty in each bin is of the order of $\sim20\,\%$ \citep[using the adaptive binning introduced by][]{lott2012}, are shown in Fig.~\ref{fig:lc_minutes}. 
In order to make an unbiased selection of GTIs that we want to test against the hypothesis of a constant flux, we consider only those GTIs, where the BBs indicate a significant flux change within the GTI (GTIs marked in green in Fig.~\ref{fig:lc_minutes}.
Naively, one would think that a BB change within one GTI would correspond to a significance of 95\,\% for a non-constant flux, since this is the threshold we have selected in the BB algorithm~\citep{2013ApJ...764..167S}. However, the BBs take also data  before and after the particular GTI into account and only provide qualitative evidence for minute-scale variability. 
Therefore, we test each bin selected in this way against the hypothesis of constant flux, using a simple $\chi^2$ test. 
The best-fit constant flux is given by $\hat{F} = (\sum (F_i / \sigma_i^2))(\sum F_i^{-2})^{-1}$. 
The pre- and post-trial $p$-values of the $\chi^2$ fits are provided in Tab.~\ref{tab:minute} which have a pre-trial probability of less than 0.1 for a constant flux. We count each tested GTI for each flare as one trial. 
We also provide the minimum values for the variability times for pairs of fluxes $F_i$ and $F_j$ measured at $t_i$ and $t_j$, respectively, given by~\citet{1999ApJ...527..719Z},
\begin{equation}
t_{\mathrm{var},ij} = \frac{F_i + F_j}{2}\left|\frac{t_i - t_j}{F_i - F_j}\right|.
\end{equation}
The pre-trial $p$-values for rejecting the constant-flux hypothesis are all around to $2\,\sigma$. 
The trial correction leaves only one GTI of 3C279 and two GTIs for CTA102 close or above a $2\,\sigma$ evidence for a variable flux. 
For these GTIs the suggested variability times scales are between 3 and 8\,minutes.

In comparison to previous results for 3C279 and CTA102, our method results in lower significance detection of minute-scale variability. 
This is due to trial correction but probably also due to differences in the analysis. For example, we use a finer binning for the exposure in the azimuthal direction to take this dependence for such short observations into account. 
The change in exposure is, however, below 10\,\%.
More importantly, we use a different binning within one GTI, which can also change the significance. 
\todo{Check other results. Can we say that we cannot confirm the high significance of Shukla et al? Their bin included?} 
Taken $\chi^2$ test and BBs together, we find weak evidence for minute-scale variability for 3C454.3 and PKS1510-089 and stronger evidence for 3C279 and CTA102.
\begin{figure*}
    \centering
    \includegraphics[width = .9\linewidth]{figures/lc_minute_3min.pdf}
    \caption{Sub-GTI light curves for the HOP groups of the orbital light curves which contain one bin with $\mathrm{TS} \geqslant 150$. The green shaded regions indicate GTIs that encompass a significant flux change found by the BB algorithm. The dark grey curves show the relative exposure at 1\,GeV \todo{check}. The ToO campaigns for 3C279 (upper left panel) and CTA102 (upper right panel) are clearly visible as the exposure is constant over several GTIs. }
    \label{fig:lc_minutes}
\end{figure*}

\begin{deluxetable}{cccccc}
\tablewidth{0pt}
\tabletypesize{\scriptsize}
\tablecaption{ \label{tab:minute}Results from sub-GTI light curves on minutes-scale variability.}
\tablehead{$t_0$ & $\Delta t$  & $\chi^2 / \mathrm{d.o.f.}$ & $p$-value & $p$-value& $\mathrm{min}(t_\mathrm{var})$  \\ 
{} [MJD] & [mins] & {} & {} & (post trial) & [mins]}
\startdata
\hline
\multicolumn{6}{c}{3C279}\\
\hline
57189.07 & 30.72 & 1.93 & 0.051 (1.95$\sigma$) & 0.188 (1.32$\sigma$) & $5.6\pm2.8$ \\
57189.14 & 35.13 & 1.68 & 0.071 (1.81$\sigma$) & 0.254 (1.14$\sigma$) & $3.6\pm1.4$ \\
57189.47 & 53.08 & 1.94 & 0.015 (2.42$\sigma$) & 0.060 (1.88$\sigma$) & $3.7\pm1.4$ \\
\hline
\multicolumn{6}{c}{PKS1510-089}\\
\hline
55854.07 & 50.80 & 2.01 & 0.091 (1.69$\sigma$) & 0.173 (1.36$\sigma$) & $7.9\pm5.0$ \\
\hline
\multicolumn{6}{c}{CTA102}\\
\hline
57738.07 & 37.04 & 2.30 & 0.011 (2.55$\sigma$) & 0.032 (2.14$\sigma$) & $2.8\pm1.0$ \\
57758.86 & 78.00 & 2.62 & 0.049 (1.97$\sigma$) & 0.049 (1.97$\sigma$) & $7.8\pm3.7$ \\
\hline
\multicolumn{6}{c}{3C454.3}\\
\hline
55520.25 & 25.83 & 1.96 & 0.048 (1.98$\sigma$) & 0.216 (1.24$\sigma$) & $3.2\pm1.6$ \\
\enddata
{
\tablecomments{The number of trials is counted for each flare individually and given by the number of green shaded areas in each panel of Fig.~\ref{fig:lc_minutes}.}
}
\end{deluxetable}


\section{Location of the $\gamma$-ray emitting region}
The location of the \gray emission region in blazar jets remains unknown. 
From the rich data set of six FSRQs studied here, we attempt to constrain the position of the emitting region using two independent approaches. 
First, in Sec.~\ref{sec:blrabs}, we search for absorption signatures in the LAT spectra caused by pair production of \Grays with photons of external photon fields. We use spectra during the brightest flares identified in the orbital light curves in Fig.~\ref{fig:gti}.
The derived constraints are used to calculate energy dependent cooling times in Sec.~\ref{sec:tcool}, which will be compared against our results for the decay times for the whole energy range (see Sec.~\ref{sec:results-local}) and for energy dependent light curves.
As shown by~\citet{2012ApJ...758L..15D}, if the flux decay is dominated by radiative cooling in external radiation fields, the energy dependence of the decay times can be used to distinguish inverse-Compton cooling in the radiation fields of the BLR or the dust torus.
This provides additional information about the position of the \gray emission region.
Second, in Sec.~\ref{sec:gammaradio}, we search for time lags between \gray and radio emission. 
In the scenario where the non-thermal emission is triggered by, e.g., shocks propagating downstream through the jet, a time lag can be translated into the spatial separation between the radio and \gray emitting regions~\citep{2014MNRAS.445..428M}. 
With information about the location of the radio core, the position of the \gray emitting region can be constrained as well~\citep[e.g.,][]{2014MNRAS.441.1899F}. 

\subsection{Results from spectral fits to \gray data}
\label{sec:blrabs}
The attenuation due to pair production on a radiation field of soft photons should manifest itself as a cut-off feature in the \gray spectrum. 
The cut-off energy depends on the distance of the \gray emitting region to the central black hole, $r$, and the photon density of the considered photon field.
For FSRQs, photon densities of external radiation fields of the accretion disk, the BLR, and the extended dust torus usually dominate the one of internal synchrotron emission~\citep[see,e.g.,][]{2012ApJ...758L..15D}.
The most relevant external photon field for the \gray energies which can be probed with \FermiLAT is the BLR. 
Pair production on photons from the dust torus or the accretion disk only becomes important at energies beyond 1\,TeV, even when the \Grays are produced closed to the central black hole~\citep{finke2016}.

We therefore search for BLR absorption features by fitting the observed \gray spectra during the brightest flares with functions the form 
\begin{eqnarray}
    f(E,\vec{\pi},r,z) &=& f_\mathrm{int}(E,\vec{\pi}) \times \nonumber \\ &{}& \exp\left[-\left(\tau_{\gamma\gamma}^\mathrm{BLR}(E,r) +\tau_{\gamma\gamma}^\mathrm{EBL}(E,z) \right)  \right],
\end{eqnarray} 
where $f_\mathrm{int}(E, \vec{\pi})$ describes the intrinsic spectrum at \gray energy $E$ emitted by the source, which depends on spectral parameters, $\vec{\pi}$, and $\tau_{\gamma\gamma}^\mathrm{BLR / EBL}$ is the optical depth due to interactions of \Grays with BLR and EBL photons, respectively. 
For the EBL optical depth, which depends on $E$ and the source redshift, $z$, we use the EBL model of \citet{2011MNRAS.410.2556D}.
The BLR optical depth will depend on the \gray energy, the location of the \gray emission zone, and the BLR geometry. 
Here, we follow the stratified BLR model introduced by \citet{finke2016}, who models the BLR either as a collection of shells or rings perpendicular to the jet axis, in order to emulate a flattened BLR. 
Each shell or ring is assumed to have infinitesimal thickness and to emit a monochromatic UV or optical emission line. 
The radii $R_\mathrm{li}$ of the shells and rings as well as the line luminosities $L_\mathrm{li} = \xi_\mathrm{li}L_\mathrm{disk}$ are taken from templates of average spectra obtained in reverberation mapping campaigns relative to the radius and luminosity of the H$\beta$ line~\citep[see][for further details]{finke2016}.
With the H$\beta$ luminosities listed in Tab.~\ref{tab:src-select}, we fix the absolute luminosities (or conversely $\xi_{\mathrm{H}\beta}$) and radii of all lines included in the model.  
Together with the masses of the super-massive black holes, we can then calculate $\tau_{\gamma\gamma}^\mathrm{BLR}$ for both geometries as a function of $r$ and observed \gray energy $E$.
In the BLR model, the absorption is dominated by pair production with Ly$\alpha$ photons at rest-frame energy of $\epsilon_{\mathrm{Ly}\alpha}\sim10.2\,$eV emitted at radii between $\sim 8\times10^{16}$ and $2\times10^{17}$\,cm. 
In the ring geometry, the corresponding energy density, which is assumed to be isotropic in the stationary frame of the galaxy, becomes~\citep{finke2016}
    \begin{equation}
        u_\mathrm{BLR} \approx u_{\mathrm{Ly}\alpha}= \frac{\xi_{\mathrm{Ly}\alpha}L_\mathrm{disk}}{4\pi c(R_{\mathrm{Ly}\alpha}^2 + r^2)},
        \label{eq:u-blr}
    \end{equation}
and takes values $\sim 5\times10^{-2}\,\mathrm{erg}\,\mathrm{cm}^{-3}$ regardless of the source for $r = R_{\mathrm{Ly}\alpha}$.
These numbers can be compared against typical values for the BLR radius, $R_\mathrm{BLR} \sim 10^{17}\,\mathrm{cm}\, (L_\mathrm{disk} / 10^{45} \mathrm{erg}\,\mathrm{s}^{-1})^{1/2}$ \citep[e.g.][]{2007ApJ...659..997K,2009ApJ...697..160B} and energy density $u_\mathrm{BLR} \sim 10^{-2}\,\mathrm{erg}\,\mathrm{cm}^{-3} $ (again in the stationary galaxy frame) assuming  $L_\mathrm{BLR} = \xi_\mathrm{BLR} L_\mathrm{disk}$ with $\xi_\mathrm{BLR}\sim 0.1$. 
The chosen BLR model gives values broadly consistent with typical values within a factor of a few.

Typically, FSRQ spectra show intrinsinc curvature, even below energies at which BLR absorption becomes important (see, e.g., the 3FGL). Therefore we chose a log-parabola for the intrinsic spectral function and also test a power law with super-exponential cut-off. 
In the fit, we only include energy bins above 1\,GeV, as we expect the BLR cut-off at energies $\gtrsim 10\,$GeV. In this way, we avoid that the best fit is determined mainly by the high photon statistics below 1\,GeV.
Additionally, we select narrow time intervals around the brightest flares (see Sec.~\ref{sec:zoom} and Fig.~\ref{fig:gti}).
This is a compromise between sufficient photon statistics to probe energies above 10\,GeV and avoiding the mixing of different activity states with potentially different spectral states. 
From Fig.~\ref{fig:specvar} we see that for the highest fluxes only marginal spectral variability is present, which should render our results robust against potential variations of the intrinsic spectra. 
Also, we only include flares which have energy bins detected with $\mathrm{TS} > 0$ where the absorption exceeds 80\,\% for the smallest BLR distance tested ($r = 10^{-2}R_{\mathrm{Ly}\alpha})$. 
We skip a flare if this is not the case since no strong limits on $r$ can be derived in this case.
This excludes all flaring periods from 3C273, for which we cannot obtain any limits from the \gray spectra.

We derive the best-fit values for the spectral parameters $\vec{\pi}$ and the distance $r$ with a likelihood maximization of the bin-by-bin likelihood curves using \textsc{Minuit} \citep{},\footnote{The bin-by-bin likelihoods are derived by fixing the spectral shape in each bin to a power law and mapping the likelihood as a function of the normalization. In the process, the spectral parameters of the neighbouring point sources and diffuse backgrounds are fixed to their broad band best-fit values.} which we extract with \textsc{fermipy} and that are shown as gray shaded bands in the panels of Fig.~\ref{fig:seds}. 
The flux points in the figure coincide with the maximum likelihood. 
Also shown are the best-fit spectra and BLR attenuation for different distances $r$ (colored curves). 
We only include energy bins and their likelihoods if the bin-by-bin fit converged and if the source is detected with $\mathrm{TS} > 0$.\footnote{We do not have to limit ourselves to bins with even larger $\mathrm{TS}$, since our bin-by-bin likelihood approach takes full Poisson statistics into account. \todo{formulate a bit clerarer.}}


\begin{figure*}
    \centering
    \begin{tabular}{ccc}
    \includegraphics[width=0.32\linewidth]{figures/sed_PKSB1222+216_t001_LogParabola_3min_ring_emin1000.pdf} &
    \includegraphics[width=0.32\linewidth]{figures/sed_3C279_t001_LogParabola_3min_ring_emin1000.pdf} & 
    \includegraphics[width=0.32\linewidth]{figures/sed_3C279_t003_LogParabola_3min_ring_emin1000.pdf}\\
    \includegraphics[width=0.32\linewidth]{figures/sed_PKS1510-089_t005_LogParabola_3min_ring_emin1000.pdf} &
    \includegraphics[width=0.32\linewidth]{figures/sed_PKS1510-089_t006_LogParabola_3min_ring_emin1000.pdf} & 
    \includegraphics[width=0.32\linewidth]{figures/sed_3C454p3_t001_LogParabola_3min_ring_emin1000.pdf}}\\
    \includegraphics[width=0.32\linewidth]{figures/sed_CTA102_t002_LogParabola_3min_ring_emin1000.pdf} & 
    \includegraphics[width=0.32\linewidth]{figures/sed_CTA102_t003_LogParabola_3min_ring_emin1000.pdf} & 
    \includegraphics[width=0.32\linewidth]{figures/sed_CTA102_t004_LogParabola_3min_ring_emin1000.pdf}\\
    \includegraphics[width=0.32\linewidth]{figures/sed_CTA102_t001_LogParabola_3min_ring_emin1000.pdf}
    \end{tabular}

    \caption{Log-parabola fits above 1\,GeV to bright \gray flares detected at energies that correspond to a attenuation in the BLR of at least 20\,\% (for $r = 10^{-2}R_{\mathrm{Ly}\alpha}$). The attenuation due to the interactions with BLR photons (assuming the ring geometry) is shown as colored lines. The best fit (95\,\% lower limit on $r$) is shown as a black dashed (dash-dotted) line. 
    The fit uses the bin-by-bin likelihood curves shown as gray bands. The numbers below and above the flux points show the \mathrm{TS} values  with which each bin is detected and the number of \Grays associated with the source at a probability $>85\,\%$, respectively.}
    \label{fig:seds}
\end{figure*}

For both tested BLR geometries, the best-fit value of $r$ is always close to or coincides with the maximum tested value, $r = 10R_{\mathrm{Ly}\alpha}$, and hence no significant absorption is found. found (dashed black lines in Fig.~\ref{fig:seds}). Consequently,  we use \textsc{Minos} to derive the profile likelihood as a function of $r$ from which we determine the 95\,\% lower limit on $r$ (dash-dotted black lines). 
The limit values are reported for each flare in Fig.~\ref{fig:blr_limits} and  summarized in Tab.~\ref{tab:blrabs} for the ring BLR geometry and log-parabola spectrum.
Assuming instead a power law with super-exponential cut-off yields consistent results. 
For the BLR shell geometry the lower limits are a factor of $\sim2$-$3$ higher because this geometry predicts stronger absorption~\citep{finke2016}. The ring geometry is therefore the conservative choice. 

As can be seen from Tab.~\ref{tab:blrabs} and Fig.~\ref{fig:blr_limits} the limits are of the order of $r_\mathrm{lim}\sim10^{17}$cm which translates to a distance close to or even beyond the Ly$\alpha$ emitting ring and consequently the BLR itself. In terms of gravitational radii, the emission regions should be located at distances of at least $\sim10^3r_g$. 
Table~\ref{tab:blrabs} also reports the energy of the highest energy photon (HEP) associated with the FSRQ with at least $99\,\%$ probability. For all but one source, this energy is larger than the energy where the optical depth due to absorption in the BLR exceeds $t_{\gamma\gamm}^\mathrm{BLR} > 1$ (assuming $r = r_\mathrm{lim}$). 

The limits depend sensitively on the detection of the source at energies as large as possible. To ensure that the detections are indeed not spuriuos, we also report the detection significance and the number of detected \Grays (associated with the source with a probability $>85\,\%$) for each energy bin below and above the flux points in Fig.~\ref{fig:seds}, respectively. The highest energy bins only contain a handful of source photons (1-4), which underlines the necessity to use the full Poisson likelihood information. 
Nevertheless, the source detections in these energy bins correspond to significances (approximated with  $\sqrt{\mathrm{TS}}$) $\gtrsim 4\,\sigma$.
The reason is that in the considered energies and short time spans (see Tab.~\ref{tab:blrabs}) the number of expected background events is small. 
\todo{comparison with costamante and others?}

\begin{figure}
    \centering
    \includegraphics[width = .9\linewidth]{figures/limits.pdf}
    \caption{Lower limits on the distance $r$ of the \gray emitting region to the central black hole. Limits from fits to \gray spectra are shown as diamonds, values derived from variability considerations are shown as bullets. }
    \label{fig:blr_limits}
\end{figure}

\begin{deluxetable*}{ccccccc|ccc}
\tablewidth{0pt}
\tablecaption{ \label{tab:blrabs}Results from BLR absorption fits to \gray spectra.}
\tablehead{$t_0$ & $\Delta t$  & $r_\mathrm{limit}$ & 
$r_\mathrm{limit}$ & $r_\mathrm{limit}$  & $E_\mathrm{HEP}$  & $E_{\tau_{\gamma\gamma} = 1}$ & $t_\mathrm{cool,~BLR}$  &
$t_\mathrm{cool,~dt}$  & $\tau_\mathrm{decay}$\\
{} [MJD] & [days] & $[10^{17}\mathrm{cm}]$ & $[R_{\mathrm{Ly}\alpha}]$ & $[r_g]$ & [GeV] & [GeV] &  [mins] & [hours] & [hours]
}
\startdata
\hline
\multicolumn{10}{c}{PKSB1222+216}\\
\hline
55364.68 & 3.42 & 1.33 & 1.40 & 609 & 75.39 & 69.69 & 8.2 & 26.8 & $47.4\pm8.3$\\
\hline
\multicolumn{10}{c}{3C279}\\
\hline
57188.07 & 1.87 & 0.49 & 0.64 & 867 & 56.03 & 42.91 &  2.7 & 19.0 & $0.5\pm0.9$\\
58133.34 & 5.32 & 1.45 & 1.91 & 2580 & 92.56 & 107.91& 9.0 & 19.0 & $8.2\pm6.3$\\
\hline
\multicolumn{10}{c}{PKS1510-089}\\
\hline
57114.16 & 1.42 & 0.51 & 0.66 & 1088 & 66.54 & 54.99 & 0.6 & 4.5 & $0.4\pm0.3$\\
57243.84 & 4.53 & 0.74 & 0.97 & 1591 & 75.93 & 65.39 & 0.8 & 4.5 & $44.4\pm9.4$\\
\hline
\multicolumn{10}{c}{CTA102}\\
\hline
57737.41 & 1.67 & 1.41 & 0.86 & 562 & 36.25 & 21.23 & 1.0 & 6.4 & $0.3\pm0.5$\\
57749.10 & 4.99 & 3.20 & 1.95 & 1275 & 73.80 & 37.94& 2.8 & 6.4 & $8.7\pm1.2$ \\
57757.55 & 4.66 & 2.76 & 1.67 & 1096 & 39.19 & 32.38& 2.2 & 6.4 & $24.6\pm2.3$ \\
57861.71 & 2.42 & 1.95 & 1.18 & 776 & 34.73 & 24.94& 1.4 & 6.4 & $1.2\pm0.7$ \\
\hline
\multicolumn{10}{c}{3C454.3}\\
\hline
55516.55 & 8.93 & 3.19 & 1.36 & 1598 & 41.19 & 28.73& 4.2 & 16.8 & $2.6\pm1.0$ \\
\enddata
{
\tablecomments{Highest energy photons are given for source probabilities $> 0.99$. Decay time given for flare component with highest peak flux.}
}
\end{deluxetable*}

We also compare the limits from fits to \gray spectra to considerations from variability arguments in Fig.~\ref{fig:blr_limits}.
If the emission region $R'_\mathrm{blob}$ (the prime denotes the co-moving frame) is causally connected during the flare, the shortest variability time $t_\mathrm{var}$ set a lower limit on its size \citep[e.g.,][]{2008MNRAS.384L..19B}:
\begin{equation}
    R'_\mathrm{blob} \leqslant \frac{ct_\mathrm{var}\delta_\mathrm{D}}{1+z},  
\end{equation}
where $\delta_\mathrm{D} = \Gamma^{-1}(1 + \beta\cos\theta_\mathrm{obs})^{-1}$ is the Doppler boost factor with the bulk Lorentz factor of the flow, $\Gamma$, $\beta = \sqrt{1 - \Gamma^{-2}}$ the associated velocity, and $\theta_\mathrm{obs} $ is the angle between the line of sight and the jet axis.
Under the assumption of a conical jet with opening angle $\theta_\mathrm{j} \sim \Gamma^{-1}$, we obtain an upper limit on the distance to the black hole, $r \sim R'_\mathrm{blob}\Gamma \equiv r_\mathrm{j}$.
The values are plotted in Fig.~\ref{fig:blr_limits} for the minimum of the rise and decay times of the brightest flares found in Fig.~\ref{fig:gti}.
The average values for $\delta_\mathrm{D}$ and $\Gamma$ obtained from Very Long Baseline Array monitoring observations are used \citep{2017ApJ...846...98J}  and the total  uncertainty is obtained by summing the uncertainties on $\delta_\mathrm{D}$, $\Gamma$, and the fit uncertainty of $t_\mathrm{var}$ in quadrature.
In general, we find that $r_\mathrm{lim} \gtrsim r_\mathrm{j}$ indicating that the emission regions are at larger distances to the black hole than than predicted from the conical jet scenario.  

\subsection{Considerations from radiative cooling}
With the limits on $r$ it is now possible to derive the energy density of the external photon field in the co-moving frame, which could be responsible for the \gray emission due to inverse-Compton (IC) scattering with relativistic electrons in the emission region. 
This further enables us to compare the predicted IC cooling times with the observed decay times. 
In the galaxy frame, the energy density of the BLR in the ring geometry is approximately given by Eq.~\ref{eq:u-blr}, and hence the photon number density is $n_\mathrm{BLR} \approx u_\mathrm{BLR} / \epsilon_{\mathrm{Ly}\alpha}$.
Assuming that the BLR photon field is isotropic and in the limit $\Gamma \gg 1$, the energy density in the co-moving frame becomes:
$u'_\mathrm{BLR} = (4/3)\Gamma^2 u_\mathrm{BLR}$~\citep{1994ApJS...90..945D,2002ApJ...575..667D}.
We calculate the energy loss of the electrons, $\dot{\gamma}_\mathrm{BLR}'$, due to IC scattering in the co-moving frame numerically, in order to incorporate Klein-Nishina   effects following \citet{1970RvMP...42..237B}.
The observed cooling time is then given by
\begin{equation}
    t_\mathrm{cool,BLR} = \frac{1 + z}{\delta_\mathrm{D}} \frac{\gamma'}{\dot{\gamma}'_\mathrm{BLR}}.
\end{equation}
In the Thomson regime, this becomes
\begin{equation}
    t_\mathrm{cool,BLR} = \frac{1+z}{\delta_\mathrm{D}}\frac{3m_ec^2}{4c\sigma_\mathrm{T}u_\mathrm{BLR}'\gamma_\mathrm{BLR}'},
    \label{eq:tcool-thomson}
\end{equation}
where $m_e$ is the electron mass and $\sigma_\mathrm{T}$ the Thomson cross section. 
In what follows, we approximate the electron Lorentz factor with~\citep[e.g.,][]{2009herb.book.....D,finke2016}
\begin{equation}
    \gamma'_\mathrm{BLR/dt} = \frac{1}{ \delta_\mathrm{D}}\sqrt{\frac{E(1+z)}{2\epsilon_{\mathrm{Ly}\alpha}}},
\end{equation}
where $E$ is the observed \gray energy. 
From Eq.~\ref{eq:tcool-thomson} and \ref{eq:u-blr} it becomes clear that the cooling time scales as $t_\mathrm{cool, BLR}\propto r^2 \Gamma^{-2}$, i.e., the cooling becomes less efficient for large distances. 
Furthermore, if the \gray emission is produced very far away from the BLR, the external photons will appear as a point source illuminating the emission region from behind, so that $u'_\mathrm{BLR} = (1/4)\Gamma^{-2} u_\mathrm{BLR}$,~\citep{1994ApJS...90..945D} leading to an additional decrease of the cooling time with $\Gamma^2$.

As an example, we compare the predicted cooling times against the observed cooling time from the fit to the orbital light curve in Fig.~\ref{fig:tcool}.

\begin{figure}
    \centering
    \includegraphics[width = .9\linewidth]{figures/tcool_CTA102_t003_LogParabola_ring.pdf}
    \caption{Cooling times for IC scattering with BLR photons and photons from the dust torus. Also shown are observed decay times for the full energy range, energy dependent light curves, and sub-orbital light curves. \todo{formulate better, add description of second x-axis}}
    \label{fig:tcool}
\end{figure}

\label{sec:tcool}
    \begin{itemize}
  \item 
As noted by \citet{2012ApJ...758L..15D}, the energy dependence of the cooling times could reveal the dominant photon field responsible for IC scattering:  Now discuss energy dependent light curves, see my notebook (on source by source basis, see also below), then discuss data points in figure
\end{itemize}



We split the data into three energy bins, from 100\,MeV-300\,MeV, 300\,MeV-1\,GeV, and 1\,GeV-100\,GeV and recompute the orbital light curves in Fig.~\ref{fig:gti}. 
The energy bins are chosen as a compromise between number of bins and sufficient photon statistics in each bin. 
The light curves for which at least 2 BBs are identified in each energy bin are shown in Fig.~\ref{fig:lcebins}.


\begin{figure*}
    \centering
    \includegraphics[width = .9 \linewidth]{figures/lc_ebins_ts9.pdf}
    \caption{Caption}
    \label{fig:lcebins}
\end{figure*}

\begin{figure*}
    \centering
    \includegraphics[width = .9 \linewidth]{figures/zdcf_ebins.pdf}
    \caption{\todo{If $\tau > 0$: second (high energy) LC lags behind first (low energy) curve. If $\tau < 0$, high energy curve (second LC) is leading}}
    \label{fig:zdcf}
\end{figure*}



\subsection{Results from radio-\gray correlation analysis}
\label{sec:gammaradio}

\todo{
Things done differently with respect to Fermi periodograms:
\begin{itemize}
    \item Used variance matching instead of EM13 algorithm as it did not reprocude the observed periodograms, however the variance matching did. Also, PSDs generally closer or above two, Parseval's theorem applies
    \item no uncertainties of light curve applied, generally led to a strong flattening of simulated PSDs for high beta values, not observed in observed PSD. Selection of errors not perfect / too high, that's why white noise at high frequencies, also correlation between errors and fluxes observed, not taken into account when drawing randomly
    \item Radio light curves show large gaps, light curve is split when gap is larger than 20 times the median gap (for OVRO and SMA) and 4.5 times (for ALMA Band 3) between observations. Compromise between too many splits and too short individual light curves and interpolating over too large gaps. Interpolation step chosen to be 80\,\% quantile of the gaps in the split light curves, again as compromise between inventing data points and being able to sample a broad frequency range
    \item individual light curves log averaged following \citet{1993MNRAS.261..612P} paper.
\end{itemize}
}
\todo{Computing LCCF and time delay between radio and \gray light curves:
}

\begin{figure*}
    \centering
    \includegraphics[width = .9\linewidth]{figures/lc_gamma_radio_tsmin9.pdf}
    \caption{Radio light curves \todo{update figure: normalize to max flux in each band and check that time span is the same}}
    \label{fig:lc-radio}
\end{figure*}


\begin{itemize}
    \item short intro on radio data and that we use OVRO, ALMA, SMA. Light curves in Fig.~\ref{fig:lc-radio}
    \item Followed \citet{2014MNRAS.445..437M,2014MNRAS.445..428M} to compute LCCF
    \item did so for OVRO at 15\,GHz, ALMA Band 3 (3.6-2.6mm or 84-116 GHz) and SMA 1.3 mm (350 GHz)
    \item Used PSDs determined for radio and \gray data to calculate significance of LCCF: We use 5000 pairs of simulated light curves to calculate confidence bands on LCCF of 68\,\%, 95\,\%, and 99\,\% confidence level. 
    \item bins of time lags: minimum and maximum of tested time lags chosen to be half the length of the shortest light curve
    \item For binning used maximum of half the median of time separations of data points in light curve
    \item LCCF curves are shown in Fig.
    \item determined peak of time lag $\tau_\mathrm{peak}$, and give its significance if $p_\tau$-value $<0.05$, if $\tau < 0$, it means that \gray light curve is leading the radio light curve, considered only peaks with a delay $\tau < 100$ days as we expect the \gray emission to lead radio emission, as found in, e.g.,~\citet{2014MNRAS.441.1899F}
    \item Compare to 2018 paper of Max-Moerbeck et al. who also studied optical!
    \item Uncertainty of $\tau_\mathrm{peak}$ calculated with flux randomization and random subsample selection (drawing 1000 samples) following \citet{1998PASP..110..660P} as suggested by \citet{2014MNRAS.445..437M}
    \item distance between \gray and radio emitting region determined through
    \begin{equation}
        d_{\gamma,r} = \frac{\Gamma\delta\beta c\tau_\mathrm{peak}}{1 + z}
    \end{equation},
    where $\Gamma$ is the Lorentz factor, $\delta$ the Doppler-boost factor, and $\beta = \sqrt{1 - \Gamma^{-2}}$ the velocity of the flow. 
    \item We use the average Doppler-boost and Lorentz factor and their uncertainties as determined from VLBA observations from 2007 to 2013 at 43\,GHz~\citep{2017ApJ...846...98J}
    \item discussion: 3C454.3: ALMA and SMA data consistent with now time delay taking uncertainties into account. Delay for 15 GHz, expected since core further downstream in jet. Most significant detection in SMA 1.3mm data with significance $2\times10^{-4}$ ($3.72\sigma$); 3C279: radio light curves either do not correlate (OVRO) or have poor sampling during the major outbreaks. Hence, to correlation is observed. Similarly for PKS1510-089. 
    \item Following \citet{2014MNRAS.441.1899F}, we can use VLBI data to determine the position of the radio core and hence the position of the \gray core using the core shift technique: \todo{short description}
    From the MOJAVE blazar monitoring program the \citep{2012A&A...545A.113P}, the following distances $d_\mathrm{core,~15GHz}$ were determined at a frequency of 15 GHz under the assumption that $k_r = 1$: PKSB1222+216: 23.41 pc, 3C279: $<7.88$ pc, PKS1510-089: 17.71 pc,
    CTA102: 46.7 pc, and 3C454.3: 20.36 pc. Dedicated analysis have also been carried out and found for 3C454.3 $k_r = 0.6-0.8$ and $d_\mathrm{core,~15GHz} \sim 38$pc and, since $d_{\mathrm{core},\nu}\propto\nu^{-1/k_r}$,  $d_\mathrm{core,~43GHz} \sim 9$pc~\citep{2014MNRAS.437.3396K}. 3C273: \citet{2013ARep...57...34V} find $k_r = 1.4$ and $d_{\mathrm{core},\nu} = 134\nu^{-1/1.4}$ using radio observations at frequencies between 4.8 and 362 GHz.
    CTA102: \citet{2015A&A...576A..43F} conducted VLBA observations ranging from 5 GHz to 86 GHz and find $k_r = 1.0$ as a best fit value and find $d_{\mathrm{core,~86GHz}}\sim7$pc.
    \item radio/radio correlation: 3C279 in ALMA/OVRO: correlation found, however, uncertainties on peak large and LCCF $\gtrsim 0.7$ for $\tau < 0$, can be explained by ALMA light curve that only starts in 2012. Other results with small $p_\tau$ values given in Table~\ref{tab:lccf-radio}
    \item calculation of $d_{\mathrm{core},r_1}$ follows \citet{2014MNRAS.441.1899F}, using radio/radio correlations and  proper motion $\mu$ taken from 15\,GHz VLBI observations \citep{2016AJ....152...12L}, so that $\Delta r_\mathrm{mas} = \mu \tau_\mathrm{peak}$, from which we calculate $\Omega_r$ and $r_\mathrm{core}$ \citep{1998A&A...330...79L} with observation angle from \citet{2017ApJ...846...98J}, Values of $k$ taken from citations above, for $3C279$ we take $k = 1$. Obtained values for 3C273 and 3C454.3 are compatible with Kutkin and Vol'vach. 
    \item Add more discussion on results. Basic findings: \Grays produced co-spatially with mm emisison, especially interesting for 3C454.3 with 0.8 pc from radio and 0.2 pc from \gray data.
\end{itemize}

\begin{figure*}
    \centering
    \includegraphics[width = .9\linewidth]{figures/lccf__nsim5000_fermi_EM13gaps-data_ovro_MM14gaps-none_lccf.pdf}
    \caption{Caption}
    \label{fig:lccf-ovro}
\end{figure*}

\begin{figure}
    \centering
    \includegraphics[width = .99\linewidth]{figures/lccf__nsim5000_fermi_EM13gaps-data_alma-band3_MM14gaps-none_lccf.pdf}
    \caption{Caption}
    \label{fig:lccf-alma}
\end{figure}

\begin{figure*}
    \centering
    \includegraphics[width = .9\linewidth]{figures/lccf__nsim5000_fermi_EM13gaps-data_sma-1p3mm_MM14gaps-none_lccf.pdf}
    \caption{Caption}
    \label{fig:lccf-sma}
\end{figure*}

\section{Discussion}

\subsection{Comparison to theory}
\todo{
\begin{itemize}
    \item see if radio delay, limits on emission region, and shortest variability time scale fit together
    \item discuss time evolution of flares with Petropoulous' paper
    \item also discuss magnetic islands, geometric jet variations,...
\end{itemize}
}
\begin{deluxetable}{lccc}
\tablewidth{0pt}
\tablecaption{ \label{tab:lccf-radio}Time lags from radio/radio LCCF analysis.}
\tablehead{Source & $\tau_{r_1,r_2}$ [days] & $p_{\tau~{r_1,r_2}}$  & $d_{\mathrm{core},r_1}$ [pc]  }
\startdata
\hline
\multicolumn{4}{c}{ALMA Band 3 \& OVRO}\\
\hline
3C273 & $-161^{+72}_{-36}$ & 0.0222 & $1.5~[0.8,1.8]~\pm0.6$ \\
3C279 & $-622^{+154}_{-168}$ & 0.0036 & $8.1~[6.1,10.3]~\pm2.6$ \\
CTA102 & --- & --- & --- \\
3C454.3 & $-667^{+100}_{-30}$ & 0.0782 & $4.6~[3.9,4.8]~\pm2.6$ \\
\hline
\multicolumn{4}{c}{SMA 1.3mm \& OVRO}\\
\hline
3C273 & $-427^{+293}_{-87}$ & 0.0256 & $4.0~[1.2,4.8]~\pm1.5$ \\
3C279 & $-165^{+12}_{-125}$ & 0.0152 & $2.2~[2.0,3.8]~\pm0.7$ \\
3C454.3 & $-110^{+14}_{-0}$ & 0.0924 & $0.8~[0.7,0.8]~\pm0.4$ \\
\hline
\multicolumn{4}{c}{SMA 1.3mm \& ALMA Band 3}\\
\hline
3C273 & $1^{+9}_{-36}$ & 0.0000 & --- \\
3C279 & $-188^{+175}_{-119}$ & 0.0002 & --- \\
3C454.3 & $3^{+10}_{-20}$ & 0.0004 & --- \\
\enddata
{
\tablecomments{A comment}
}
\end{deluxetable}

\begin{deluxetable*}{lccccc}
\tablewidth{0pt}
\tablecaption{ \label{tab:lccf}Results from PSD analysis of radio and \gray and radio LCCF analysis.}
\tablehead{Source & $\hat{\beta}$ & $p_\beta$ & $\tau_\mathrm{peak}$ [days] & $p_\tau$ & $d_{\gamma, r}$ [pc]}
\startdata
\hline
\multicolumn{6}{c}{OVRO}\\
\hline
PKSB1222+216 & $1.92^{+0.39}_{-0.59}$ & 0.59 & --- & --- & ---\\
3C273 & $2.38^{+0.30}_{-0.97}$ & 0.94 & $-416.5^{+217.0}_{-140.0}$ & 0.0068 & $10.96~[5.2,14.6]~\pm4.4$\\
3C279 & $2.29^{+0.32}_{-0.94}$ & 0.71 & --- & --- & ---\\
PKS1510-089 & $1.89^{+0.45}_{-0.84}$ & 0.34 & --- & --- & ---\\
CTA102 & $2.23^{+0.26}_{-0.92}$ & 0.84 & --- & --- & ---\\
3C454.3 & $2.20^{+0.36}_{-2.20}$ & 0.40 & $-101.5^{+49.0}_{-112.0}$ & 0.0156 & $15.39~[8.0,32.4]~\pm2.8$\\
\hline
\multicolumn{6}{c}{ALMA Band 3}\\
\hline
3C273 & $2.12^{+0.40}_{-2.12}$ & 0.73 & --- & --- & ---\\
3C279 & $1.82^{+0.38}_{-0.45}$ & 0.89 & --- & --- & ---\\
CTA102 & $1.94^{+0.42}_{-1.33}$ & 0.45 & $-216.0^{+209.0}_{-11.0}$ & 0.0092 & $58.85~[1.9,61.8]~\pm7.3$\\
3C454.3 & $1.73^{+0.36}_{-0.30}$ & 0.25 & $-27.0^{+30.0}_{-30.0}$ & 0.0164 & $4.09~[-0.5,8.6]~\pm0.7$\\
\hline
\multicolumn{6}{c}{SMA 1.3mm}\\
\hline
3C273 & $1.48^{+0.40}_{-0.33}$ & 0.17 & $-122.5^{+84.0}_{-7.0}$ & 0.0088 & $3.22~[1.0,3.4]~\pm1.3$\\
3C279 & $1.61^{+0.16}_{-0.28}$ & 0.97 & --- & --- & ---\\
3C454.3 & $1.64^{+0.31}_{-1.64}$ & 0.21 & $10.5^{+21.0}_{-28.0}$ & 0.0002 & $-1.59~[-4.8,2.7]~\pm0.3$\\
\enddata
{
\tablecomments{A comment}
}
\end{deluxetable*}


\begin{appendix}
\section{Spectral evolution of orbital light curves}
\label{sec:specvar}
The spectral evolution of the orbital light curves for each source is shown in Fig.~\ref{fig:specvar}. In each time bin, a power-law spectrum with index $\Gamma$ and integrated flux $F$ is assumed. 


\begin{figure*}
    \centering
    \includegraphics[width = .9 \linewidth]{figures/lc_specvar_flare_int.pdf}
    \caption{Spectral variations of the brightest flares on orbital time scales. Light colors refer to earlier times, dark colors to later times.}
    \label{fig:specvar}
\end{figure*}
\end{appendix}

\bibliography{mainbib}

%% This command is needed to show the entire author+affilation list when
%% the collaboration and author truncation commands are used.  It has to
%% go at the end of the manuscript.
%\allauthors

%% Include this line if you are using the \added, \replaced, \deleted
%% commands to see a summary list of all changes at the end of the article.
%\listofchanges


\end{document}

% End of file `sample62.tex'.
